\documentclass{beamer}
\usepackage[utf8]{inputenc}
\usepackage[T1]{fontenc}
\usepackage[french]{babel}
\usepackage{graphicx}
\usepackage{caption}
\usepackage{subcaption}
\usepackage{hyperref}
\usepackage[french]{algorithm2e}

\SetKwInput{Donnees}{\underline{Données}}
\SetKwInput{Res}{Résultat}
\SetKwInput{Entree}{\underline{Entrées}}
\SetKwInput{Sortie}{\underline{Sorties}}

\SetKw{KwA}{à}
\SetKw{Retour}{\underline{retourner}}
\SetKwBlock{Deb}{\underline{Début}}{\underline{Fin}}
\SetKwIF{Si}{SinonSi}{Sinon}{si}{alors}{sinon si}{alors}{finsi}
\SetKwFor{Pour}{pour}{faire}{finpour}
\SetKwFor{Tq}{tant que}{faire}{fintq}
\SetKwFor{PourTous}{pour tous}{faire}{finprts}

\title{Planètes Procédurales}
\titlegraphic{
    \includegraphics[width=2cm]{UM}
    \hspace{2cm}
    \includegraphics[width=1.5cm]{fds}
}

\institute{Master 2 IMAGINE}
\author[A. Besnier]{Aurélien Besnier}
\date{\today}

\usetheme{Madrid}
\usecolortheme{crane}

\begin{document}

\AtBeginSection[ ]
{
\begin{frame}{Plan}
    \tableofcontents[currentsection,currentsubsection]
\end{frame}
}
\maketitle

\begin{frame}{Plan}
    \tableofcontents
\end{frame}

\section{Introduction}

\begin{frame}{Introduction}

\begin{block}{Objectif}
Générer des planètes procédurales et pouvoir les faire évoluer (en temps réel).
\end{block}
\onslide<2->
\begin{figure}[H]
        \begin{subfigure}[b]{0.25\textwidth}
                \centering
                \includegraphics[width=.5\textwidth]{ISO_C++_Logo.png}
        \end{subfigure}%
        \begin{subfigure}[b]{0.25\textwidth}
                \centering
                \includegraphics[width=.9\textwidth]{opengl.png}
        \end{subfigure}%
        \begin{subfigure}[b]{0.25\textwidth}
                \centering
                \includegraphics[width=.7\textwidth]{qlgviewer}
        \end{subfigure}%
        \onslide<3>
        \begin{subfigure}[b]{0.25\textwidth}
                \centering
                \includegraphics[width=.7\textwidth]{cgal_logo}
        \end{subfigure}
        \caption{Technologies utilisées}
\end{figure}
    
\end{frame}


\section{Initialisation}

\subsection{Création de la sphère}
\begin{frame}{Création de la sphère}
\begin{minipage}{0.49\textwidth}
\begin{scriptsize}

\begin{algorithm}[H]
\Donnees{goldenRatio=$(1 + 5^{0.5})/2$}
\Pour{i de 0 à $nb\_elem$}
{
        $theta\gets 2\times i  / goldenRatio$\;
        $phi\gets arccos(1-2\times(i+0.5)/nb\_elems)$\;
        $x\gets cos(theta) \times sin(phi)$\;
        $y\gets sin(theta) \times sin(phi)$\;
        $z\gets cos(phi)$\;
}
\end{algorithm}

\end{scriptsize}
\footnotesize Appliquer ensuite une triangulation de Delaunay sur le nuage de point résultant.

\vspace{50px}

\tiny{Source: \href{https://extremelearning.com.au/how-to-evenly-distribute-points-on-a-sphere-more-effectively-than-the-canonical-fibonacci-lattice/}{extremelearning.com.au}}
\end{minipage}
\begin{minipage}{0.45\textwidth}
\onslide<2>
\begin{figure}
 \centering
 \includegraphics[width=0.8\textwidth]{trig}
\end{figure}
\begin{figure}
 \centering
 \includegraphics[width=0.8\textwidth]{trig_pole}
 \caption{Visualisation des triangles du pôle}
\end{figure}

\end{minipage}

\end{frame}

\subsection{Segmentation}
\begin{frame}{Segmentation}

\begin{block}{Définition}
Découpage du maillages en sous régions pour y associer une sémantique.
\end{block}


\begin{figure}[H]
        \begin{subfigure}[b]{0.5\textwidth}
                \centering
                \includegraphics[width=.6\textwidth]{elephant_sdf_partition.png}
                \caption{Segmentation par valeurs de SDF\footnote{source: \href{https://doc.cgal.org/latest/Surface_mesh_segmentation/index.html}{CGAL}}}
        \end{subfigure}%
        \begin{subfigure}[b]{0.5\textwidth}
                \centering
                \includegraphics[width=.9\textwidth]{fig3}
                \caption{Segmentations sur sphères}
        \end{subfigure}%
        \captionof{figure}{Différentes segmentations}
\end{figure}

\end{frame}

\begin{frame}{Segmentation}
\begin{figure}
 \centering
 \includegraphics[width=0.7\textwidth]{plate-tectonics.jpg}
 \caption{Plaques tectoniques de la Terre}
\end{figure}
\tiny{Source:\url{https://geology.com/plate-tectonics.shtml}}
\end{frame}


\begin{frame}{Solution}
\begin{minipage}{0.45\textwidth}
 \begin{itemize}
  \item Calculer le 1-voisinage de chaque point.
  \item Pour chaque plaques, choisir un point de départ aléatoire.
  \item Faire un grossissement de régions de chaque plaque avec le 1-voisinage
  \item Itérer jusqu'à ce que chaque point soit assigné.
 \end{itemize}

\end{minipage}
\begin{minipage}{0.45\textwidth}
\onslide<2->
 \begin{figure}[H]
        \centering
        \includegraphics[width=0.8\textwidth]{segmentation_res}
        \caption{Résultat de segmentation avec 60000 éléments}
\end{figure}
\end{minipage}

\onslide<3>
\begin{block}{Inconvénient}
Calcul du 1-voisinage long sur de grand maillages(>60000).
\end{block}


\end{frame}

\subsection{Elevation}
\begin{frame}{Elevation}
\only<1>{
Initialisation des élévations:
\begin{scriptsize}
\begin{algorithm}[H]
\Pour{chaque plaques selon leur type}
{
        \Pour{chaque point $p$ de la plaque}
        {
                $rng\gets noise.fractal(octave, p.x + offsetX, p.y + offsetY, p.z + offsetZ)+offsetType$\;
                $elevation\gets (plateParams.plateType)\times rng$\;
                $elevation\gets (plateParams.plateType)\times rng$\;
                $mesh.vertices[p].pos \gets mesh.vertices[p].pos + ((elevation) \times mesh.vertices[p].normal)$\;
                $mesh.vertices[p].elevation\gets rng$\;
        }
}
\end{algorithm}
\end{scriptsize}
 }
\only<2>{
\begin{figure}
 \centering
 \includegraphics[width=0.5\textwidth]{init}
 \caption{Résultat initialisation avec rayon 637km, 6000 éléments, 8 octaves}
\end{figure}
}

\end{frame}


\section{Collisions}
\begin{frame}{Collisions}
Exemple collision Continent/Continent:
\begin{scriptsize}
\begin{algorithm}[H]
\Pour{chaque plaques continentales}
{
        \Pour{chaque point $p$ de la plaque}
        {
                \Si{$p$ est voisin avec une autre plaque continentale}
                {
                        $voisin \gets voisin p$\;
                        $cpy\_pos\gets p.pos + deplacement$\;
                        \Si{$distance(cpy\_pos,voisin) < distance(p,voisin)$}
                        {
                                $elevation \gets p.elevation \times 0.00013$
                                $p.elevation \gets p.elevation + elevation$\;
                                $mesh.vertices[p].pos \gets mesh.vertices[p].pos + ((elevation) \times mesh.vertices[p].normal)$\;
                        }
                }
        }
}
\end{algorithm}
\end{scriptsize}

\end{frame}



\section{Démonstration}
\begin{frame}{Démonstration}

\end{frame}

\end{document}
