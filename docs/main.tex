\documentclass{beamer}
\usepackage[utf8]{inputenc}
\usepackage[T1]{fontenc}
\usepackage[french]{babel}
\usepackage{graphicx}
\usepackage{caption}
\usepackage{subcaption}

\title{Procedural Planets}
\author{Aurelien Besnier}
\date{\today}

\usetheme{Madrid}
\usecolortheme{crane}

\begin{document}

\AtBeginSection[ ]
{
\begin{frame}{Plan}
    \tableofcontents[currentsection,currentsubsection]
\end{frame}
}
\maketitle

\begin{frame}{Plan}
    \tableofcontents
\end{frame}

\section{Introduction}

\begin{frame}{Introduction}

\begin{block}{Objectifs}

\end{block}

\begin{figure}[H]
        \begin{subfigure}[b]{0.25\textwidth}
                \centering
                \includegraphics[width=.5\textwidth]{ISO_C++_Logo.png}
        \end{subfigure}%
        \begin{subfigure}[b]{0.25\textwidth}
                \centering
                \includegraphics[width=.9\textwidth]{opengl.png}
        \end{subfigure}%
        \begin{subfigure}[b]{0.25\textwidth}
                \centering
                \includegraphics[width=.7\textwidth]{qlgviewer}
        \end{subfigure}%
        \begin{subfigure}[b]{0.25\textwidth}
                \centering
                \includegraphics[width=.7\textwidth]{cgal_logo}
        \end{subfigure}
        \captionof{figure}{Technologies utilisées}
\end{figure}
    
\end{frame}


\section{Initialisation}

\subsection{Création de la sphère}
\begin{frame}{Création de la sphère}

\end{frame}

\subsection{Segmentation}
\begin{frame}{Segmentation}

\end{frame}
\subsection{Elevation}
\begin{frame}{Elevation}

\end{frame}


\section{Collisions}
\begin{frame}{Collisions}

\end{frame}



\section{Conclusion}
\begin{frame}{Conclusion}

\end{frame}

\end{document}
